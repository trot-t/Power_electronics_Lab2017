\documentclass[a4paper,11pt]{article}
%\usepackage[T1,T2A]{fontenc}
\usepackage[utf8x]{inputenc}
%\usepackage[english,ukrainian,russian]{babel} 
\usepackage[english,russian]{babel} 
\usepackage{wrapfig}
\usepackage[table,xcdraw]{xcolor}
\usepackage{booktabs}
\usepackage{pifont}
\usepackage{graphicx}
\graphicspath{ {images/} }

\usepackage{tikz}
\usepackage{siunitx}
\usepackage[american,cuteinductors,smartlabels]{circuitikz}

\usepackage{advdate}
%\usepackage{showframe}

\setlength{\voffset}{-72pt} %отступ сверху - чтобы увидеть откомментарить \usepackage{showframe}
%\setlength{\voffset}{-36pt}
 \setlength{\footskip}{-20pt}
 \setlength{\topmargin}{0pt} 
% %\setlength{\headheight{1pt}
 \setlength{\headsep}{0pt}
\setlength{\hoffset}{-72pt}
\setlength{\marginparwidth}{0pt}
\setlength{\textwidth}{530pt}
\setlength{\textheight}{808pt} 

%\title{Список группы}
%\date{}


\newcommand*\OK{\ding{51}} % присутствовал
\newcommand*\ok{&\ding{51}} % присутствовал
\newcommand*\D{\ding{48}} % защита, defend
\newcommand*\Skip{\noindent\rule{0.3cm}{0.9pt}}


\begin{document}
%\thispagestyle{empty}
% or
\pagenumbering{gobble}
\AdvanceDate[-1] % печатаю в пятницу а нужна суббота
\begin{center}\today\end{center}
%\vspace*{1\baselineskip}
%\newcommand{\la1}{\textcyrillic{лаб.1(СИФУ)}}
%\maketitle
%\skippage
%\section{}
%\begin{table} \centering
%\begin{tabular}{@{} cl*{10}c|}% @{}}
\begin{tabular}{clccccccccc p{.4cm}|p{.4cm}|p{.4cm}|p{.4cm}|p{.4cm}}% @{}}
  \multicolumn{16}{c}{График выполнения лабораторных работ студентами 4405 группы} \\
\toprule
	& & & & & & & & & &%\rotatebox{90}{2 апреля}
%& %\rotatebox{90}{9 апреля} 
%& %\rotatebox{90}{16 апреля} 
%& %\rotatebox{90}{23 апреля} 
%& %\rotatebox{90}{30 апреля} 
%& %\rotatebox{90}{7 мая} 
%& %\rotatebox{90}{14 мая} 
%& %\rotatebox{90}{21 мая} 
%& %\rotatebox{90}{28 мая} 
%& %\rotatebox{90}{3 июня}
%& %\rotatebox{90}{4 июня}
&\multicolumn{4}{c}{{\shortstack[c]{\\защита\\ лабораторных\\работ}}} \\
\cmidrule{12-16}
&& \rotatebox{90}{\rlap{1 апреля}}
& \rotatebox{90}{\rlap{8 апреля}}
& \rotatebox{90}{\rlap{15 апреля}}
& \rotatebox{90}{\rlap{22 апреля}}
& \rotatebox{90}{\rlap{29 апреля}}
& \rotatebox{90}{\rlap{6 мая}}
& \rotatebox{90}{\rlap{13 мая}} 
& \rotatebox{90}{\rlap{20 мая}}
& \rotatebox{90}{\rlap{27 мая}}
%\rotatebox{90}{\rlap{\shortstack[c]{зачетная\\ неделя}}}
%& \rotatebox{90}{\rlap{3 июня 17:00}}
%& \rotatebox{90}{\rlap{4 июня 9:50}} 
	&\rotatebox{60}{№1}&\rotatebox{60}{№2}&\rotatebox{60}{№3}&\rotatebox{60}{№4}&\rotatebox{60}{№5} \\
\midrule
	&&\rotatebox{90}{лаб.3} & & \rotatebox{90}{лаб.5} & & \rotatebox{90}{лаб.4} &
	& \rotatebox{90}{лаб.1} & \\
	&Халявин Дмитрий	&\OK&&\OK& &\OK& &\OK&\OK& 		&5&5&4+&&5-\\
	&Смирнов Никита		&\OK&&\OK& &\OK& &\OK&\OK&$\cdot$		&5&5&4+&&5-\\
\rotatebox{90}{\rlap{~бригада №1}}       
	&Юдин Алексей		&\OK&&\OK& &\OK& &\OK&\OK&$\cdot$		&5&5&5-&&5-\\
 
\cmidrule{1-16} 
	& &\rotatebox{90}{лaб.1}&\rotatebox{90}{лaб.1}&\rotatebox{90}{лaб.4} & & \rotatebox{90}{лaб.3} & &\rotatebox{90}{лаб.2}&
	&\rotatebox{90}{лaб.5}\\ 
 		
	&Пермогорский Павел & \Skip &\OK&\OK& &\OK& &\OK&\Skip&	&?&&&&\\	
	&Трофимов Иван 	    & \Skip &\OK&\OK& &\Skip& &\Skip&Л2&		&?&&&&\\
\rotatebox{90}{\rlap{~бригада №2}}
	&Пестов Сергей	   &\OK&    & \Skip & &\OK& &\OK&Л4
&\rotatebox{90}{\rlap{\linespread{0.6}\hspace{0.cm}\scriptsize\begin{tabular}{l}попытка\\списывания\\протокола\end{tabular}}}	&&&
&\rotatebox{90}{\linespread{0.75}\hspace{-0.4cm}\tiny\begin{tabular}{l}списаны\\отчеты и\\протокол\end{tabular}}&\\
                              
\cmidrule{1-16}
	&&\rotatebox{90}{лaб.2}& & \rotatebox{90}{лaб.1} & & \rotatebox{90}{лaб.5}
& & \rotatebox{90}{лаб.3}&&\\
 		
	& Мартыненко Дмитрий	&\OK& &\OK& &\OK& &\OK& &$\cdot$	&5+&5&5&&4-\\
	& Банков Владислав	&\OK& &\OK& &\OK& &\OK&	&$\cdot$	&5-&3&5&&4-\\
	& Боцунов Александр	&\OK& &\OK& &\OK& &\OK&	&$\cdot$	&5-&3&5&&4-\\
\rotatebox{90}{\rlap{~бригада №3}}
	& Елисеев Анатолий 	&\OK& &\OK& &\OK& &\OK&	&$\cdot$	&5-&3&5&&4-\\
 
\cmidrule{1-16} 
	& &&\rotatebox{90}{лaб.5} & &\rotatebox{90}{лaб.4}&&\rotatebox{90}{лaб.2}&&\rotatebox{90}{лaб.1}&\rotatebox{90}{лaб.3}\\
 
	& Шелемба Андрей    	& &\OK& &\OK & &\OK & &\OK&$\cdot$		&5&5&&5&3-\\
	& Захаров Матвей    	& &\OK& &\OK & &\OK & &\OK&$\cdot$          	&5&5&&5&3-\\
\rotatebox{90}{\rlap{~бригада №4}}
	& Замалетдинов Илья 	& &\OK& &\OK & &\OK & &\OK&$\cdot$		&5&5&&5&3-\\
 
\cmidrule{1-16}
& &&\rotatebox{90}{лaб.2} &&\rotatebox{90}{лaб.1}&&\rotatebox{90}{лaб.4}&&\rotatebox{90}{лaб.5}\\
 
& Косьянов Никита   		& &\OK& &\OK &	&\OK& &\OK&$\cdot$		&3+&3-&&&5-\\
& Власенко Александр 		& &\OK& &\OK &	&\OK& &\OK&$\cdot$		&3+&3-&&&5-\\
& Аганеев Владимир 		& &\OK& &\OK &	&\OK& &\OK&$\cdot$	        &3+&3-&&&5-\\
 \rotatebox{90}{\rlap{~бригада №5}}
& Данилович Павел 		& &\OK& &\OK &	&\OK& &\OK&$\cdot$		&3+&3-&
&\rotatebox{90}{\rlap{\linespread{0.6}\hspace{-0.4cm}\scriptsize\begin{tabular}{l}{поделились отчетами}\\и протоколом\end{tabular}}}&5-\\
 
\cmidrule{1-16} 
& &&\rotatebox{90}{лaб.3} &&\rotatebox{90}{лaб.2}&\rotatebox{90}{лaб.2}&
\rotatebox{90}{лaб.1}&\rotatebox{90}{лaб.5}&\rotatebox{90}{лaб.4}&\\
 
& Зудиков Анатолий 		& &\OK& &\OK &	&\OK&\OK &\OK&$\cdot$ 		&4&3+&3&&4\\
& Бажин Евгений    		& &\OK& &\OK &	&\OK&\OK &\OK&$\cdot$ 		&4&3-&3&&4\\
& Голубев Максим   		& &\OK& &\Skip&\OK&\Skip&&\OK&$\cdot$		&&&&&	\\
\rotatebox{90}{\rlap{~бригада №6}}
& Ряжапов Ильнур  		& &\OK& &\Skip&\OK&\Skip&\OK&\OK&$\cdot$	&&&
&\rotatebox{90}{\rlap{\linespread{0.6}\scriptsize\begin{tabular}{l}{списаны отчеты}\\и протокол\end{tabular}}}&\\
\bottomrule
\end{tabular} 

\vskip 0.2cm
8 апреля приклеено упавшее стекло ваттметра Лаб 4.

22 апреля обнаружена сломанная ручка усиления канала II осциллографа Лаб №1.

Оценки снижены второй подгруппе за то что ни у кого не было энергетических 
характеристик.

Шелемба, Захаров, Замалетдинов -- в отчете по Лаб №4 
энергетические характеристики выпрямителя,
данные не соответствуют с протоколом. Замалетдинов переделывает отчет Лаб №4

Агантеев,Данилович,Косьянов, Власенко -- Лаб №1 - необясненная фото с осциллографа.
Власенко переделывает отчет Лаб №1, Аганеев переделывает отчет Лаб №2.

Агантеев - не знает как регистрируется ток утечки.

Юдин -- объясняет измерение реактивной мощности.

{\color{red} Шелемба -- высылает методичку по моделированию силовой электроники}

Замалетдинов показывает на стенде для Л2 улучшение энергетических характеристик с шунтирующим диодом.

Бригада №4 спрашивают у Латышко КПД по Л2.

\end{document}
